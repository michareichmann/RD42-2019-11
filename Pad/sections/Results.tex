\section{Results}
%%%%%%%%%%%%%%%%%%%%%%%%%%%%%%%%%%%%%% FRAME 0 %%%%%%%%%%%%%%%%%%%%%%%%%%%%%%%%%%%%%%%%%%%%%%%%
\subsection{Tracking Resolution}
\begin{frame}{Tracking Resolution}

	\subfigs{\subfig[.4]{.5}[r]{TR}}{\subfig[.55]{.33}{TelSchematics3}}
	
	\begin{itemize}\itemfill
		\item ROC = Plane
		\item resolution = width of the residual distribution at the plane under test
		\item can achieve \SI{\sim20}{\micro\meter} resolution at very low $\upchi^2$
		\item resolution at the front slightly better than in the back
		\begin{itemize}
			\item less multiple scattering
		\end{itemize}
% 		\item choose \SI{90}{\%} quantile to have a large number of events

	\end{itemize}
		
\end{frame}
%%%%%%%%%%%%%%%%%%%%%%%% FRAME 1 %%%%%%%%%%%%%%%%%%%%%%%%%%%%%%%
\subsection{Rate Measurements}
\begin{frame}{S129 - Mean Pulse Heights}

	\vspace*{-2ex}
	\only<1>{\fig[r]{.5}{meanPh}}
	\only<2>{\fig[r]{.5}{stds}}
	
	\begin{itemize}\itemfill
		\item every point the mean of a whole rate scan
		\item last two points have a different amplifier
		\item most points very stable over time but some fluctuate
		\item<2-> standard deviation in general below \SI{.5}{\%} of the mean pulse height
	\end{itemize}

\end{frame}
%%%%%%%%%%%%%%%%%%%%%%%% FRAME 1 %%%%%%%%%%%%%%%%%%%%%%%%%%%%%%%
\begin{frame}{S129 - Pulse Heights}

	\vspace*{-2ex}
	\fig[r]{.5}{phs}
	
	\begin{itemize}\itemfill
		\item all rate scans almost show no dependence on rate
		\item \bad{BUT:} the highest rate point is consistently lower by \SI{1}{\%}
	\end{itemize}

\end{frame}
%%%%%%%%%%%%%%%%%%%%%%%% FRAME 1 %%%%%%%%%%%%%%%%%%%%%%%%%%%%%%%
\begin{frame}{S129 - Mean Pedestals}

	\vspace*{-2ex}
	\fig[r]{.5}{meanPeds}
	
	\begin{itemize}\itemfill
		\item every point the mean pedestal whole rate scan
		\item if pedestal is not around zero \ra indicator for no voltage calibration in the DRS4 (digitiser)
		\item runs without voltage calibration match the runs with inconsistent pulse height
		\item except for scan 7 in Jul17 \ra does not mean that voltage calibration was correct
	\end{itemize}

\end{frame}
%%%%%%%%%%%%%%%%%%%%%%%% FRAME 1 %%%%%%%%%%%%%%%%%%%%%%%%%%%%%%%
\begin{frame}{S129 - Pedestals}

	\vspace*{-2ex}
	\fig[r]{.5}{peds}
	
	\begin{itemize}\itemfill
		\item very consistent behaviour for all scans
	\end{itemize}

\end{frame}
%%%%%%%%%%%%%%%%%%%%%%%% FRAME 1 %%%%%%%%%%%%%%%%%%%%%%%%%%%%%%%
\begin{frame}{B2 Rate Scans}

	\begin{minipage}[c][.7\textheight]{.34\textwidth}
		\begin{itemize}\itemfill
			\item after irradiation pulse height is very stable
			\item maximum irradiation: \SI{8e15}{\ncm}
			\item little drop for high rates at high irradiations 
			\item \ra due to decreasing signals one cut is working less efficient
			\item<2> positive and negative bias agree very well
		\end{itemize}
	\end{minipage}\hfill
	\begin{minipage}{.63\textwidth}
		\only<1>{\fig{.8}{B2All}}
		\only<2>{\fig{.8}{B2AllP}}
	\end{minipage}
	
\end{frame}
%%%%%%%%%%%%%%%%%%%%%%%% FRAME 1 %%%%%%%%%%%%%%%%%%%%%%%%%%%%%%%
\begin{frame}{B2 Pulse Height Evolution}

	\vspace*{-3ex}
	\fig[r]{.6}{B2M}

	\def\d{\hspace*{1ex}}
	\hspace*{17ex} 0 \d\ra\d \SI{5e14}{} \d\ra\d \SI{1e15}{} \d\ra\d \SI{2e15}{} \ra \SI{4e15}{} \ra \SI{8e15}{\ncm}\vspace*{3ex}
	\begin{itemize}\itemfill
		\item absolute pulse height decreases exponentially
		\item SNR at highest irradiation only 2/1 \ra prevents next step with this amplifier \ra use new OSU amp?
	\end{itemize}

\end{frame}
%%%%%%%%%%%%%%%%%%%%%%%% FRAME 3 %%%%%%%%%%%%%%%%%%%%%%%%%%%%%%%
\begin{frame}{Fix Rate Dependence}

	\subfigs{\subfig{.5}[r]{PHB}[First measurement]}{\subfig{.5}[r]{PHB2}[After reprocessing]}
	
	\begin{itemize}\itemfill
		\item less than \SI{20}{\%} of the tested diamonds show rate dependence \SI{>10}{\%}
		\item very large rate dependence at the first measurement (\SI{>90}{\%})
		\item after reprocessing and surface cleaning with RIE very stable behaviour (\SI{\sim2}{\%})
		\item feasible to ``fix'' bad diamonds
	\end{itemize}

\end{frame}
